\section{Exercise 1}

First, we use the script from database lab 4 to create and populate the database.

\begin{lstlisting}[language=sql, caption=Create database]
USE master;
CREATE DATABASE COMPANY;
GO
USE COMPANY;


CREATE TABLE EMPLOYEE (
    FNAME           VARCHAR(15)     NOT NULL,
    MINIT           CHAR,
    LNAME           VARCHAR(15)     NOT NULL,
    SSN             CHAR(9),
    BDATE           DATE,
    ADDRESS         VARCHAR(50),
    SEX             CHAR,
    SALARY          DECIMAL(10,2),
    SUPER_SSN       CHAR(9),
    DNO             INT,
    PRIMARY KEY (SSN),
    FOREIGN KEY (SUPER_SSN) REFERENCES EMPLOYEE (SSN)
);

CREATE TABLE DEPARTMENT (
    DNAME           VARCHAR(15)     NOT NULL UNIQUE,
    DNUMBER         INT,
    MGR_SSN         CHAR(9)         NOT NULL,
    MGR_START_DATE  DATE,
    PRIMARY KEY (DNUMBER)
);

CREATE TABLE DEPT_LOCATIONS (
    DNUMBER         INT,
    DLOCATION       VARCHAR(15),
    PRIMARY KEY (DNUMBER, DLOCATION),
    FOREIGN KEY (DNUMBER) REFERENCES DEPARTMENT (DNUMBER)
);

CREATE TABLE PROJECT (
    PNAME           VARCHAR(15)     NOT NULL,
    PNUMBER         INT,
    PLOCATION       VARCHAR(15)     NOT NULL,
    DNUM            INT,
    PRIMARY KEY (PNUMBER),
    FOREIGN KEY (DNUM) REFERENCES DEPARTMENT (DNUMBER)
);

CREATE TABLE WORKS_ON (
    ESSN            CHAR(9),
    PNO             INT,
    HOURS           DECIMAL(3,1),
    PRIMARY KEY (ESSN, PNO),
    FOREIGN KEY (ESSN) REFERENCES EMPLOYEE (SSN),
    FOREIGN KEY (PNO) REFERENCES PROJECT (PNUMBER)
);

CREATE TABLE DEPENDENT (
    ESSN            CHAR(9),
    DEPENDENT_NAME  VARCHAR(15),
    SEX             CHAR,
    BDATE           DATE,
    RELATIONSHIP    VARCHAR(15),
    PRIMARY KEY (ESSN, DEPENDENT_NAME),
    FOREIGN KEY (ESSN) REFERENCES EMPLOYEE (SSN)
);

ALTER TABLE EMPLOYEE
ADD FOREIGN KEY (DNO) REFERENCES DEPARTMENT (DNUMBER);

ALTER TABLE DEPARTMENT
ADD FOREIGN KEY (MGR_SSN) REFERENCES EMPLOYEE (SSN);
\end{lstlisting}

\begin{lstlisting}[language=sql, caption=Populate database]
USE COMPANY;


INSERT INTO EMPLOYEE VALUES 
    ('John', 'B', 'Smith', '123456789', '1965-01-09', '731 Fondren, Houston, TX', 'M', 30000, NULL, NULL),
    ('Franklin', 'T', 'Wong', '333445555', '1955-12-08', '638 Voss, Houston, TX', 'M', 40000, NULL, NULL),
    ('Alicia', 'J', 'Zelaya', '999887777', '1968-07-19', '3321 Castle, Spring, TX', 'F', 25000, NULL, NULL),
    ('Jennifer', 'S', 'Wallace', '987654321', '1941-06-20', '291 Berry, Bellaire, TX', 'F', 43000, NULL, NULL),
    ('Ramesh', 'K', 'Narayan', '666884444', '1962-09-15', '975 Fire Oak, Humble, TX', 'M', 38000, NULL, NULL),
    ('Joyce', 'A', 'English', '453453453', '1972-07-31', '5631 Rice, Houston, TX', 'F', 25000, NULL, NULL),
    ('Ahmad', 'V', 'Jabbar', '987987987', '1969-03-29', '980 Dallas, Houston, TX', 'M', 25000, NULL, NULL),
    ('James', 'E', 'Borg', '888665555', '1937-11-10', '450 Stone, Houston, TX', 'M', 55000, NULL, NULL);

INSERT INTO DEPARTMENT VALUES
    ('Research', 5, 333445555, '1988-05-22'),
    ('Administration', 4, 987654321, '1995-01-01'),
    ('Headquarters', 1, 888665555, '1981-06-19');


UPDATE EMPLOYEE SET SUPER_SSN = '333445555', DNO = 5 WHERE SSN = '123456789';
UPDATE EMPLOYEE SET SUPER_SSN = '888665555', DNO = 5 WHERE SSN = '333445555';
UPDATE EMPLOYEE SET SUPER_SSN = '987654321', DNO = 4 WHERE SSN = '999887777';
UPDATE EMPLOYEE SET SUPER_SSN = '888665555', DNO = 4 WHERE SSN = '987654321';
UPDATE EMPLOYEE SET SUPER_SSN = '333445555', DNO = 5 WHERE SSN = '666884444';
UPDATE EMPLOYEE SET SUPER_SSN = '333445555', DNO = 5 WHERE SSN = '453453453';
UPDATE EMPLOYEE SET SUPER_SSN = '987654321', DNO = 4 WHERE SSN = '987987987';
UPDATE EMPLOYEE SET SUPER_SSN = NULL, DNO = 1 WHERE SSN = '888665555';

INSERT INTO DEPT_LOCATIONS VALUES
    (1, 'Houston'),
    (4, 'Stafford'),
    (5, 'Bellaire'),
    (5, 'Sugarland'),
    (5, 'Houston');

INSERT INTO PROJECT VALUES
    ('ProductX', 1, 'Bellaire', 5),
    ('ProductY', 2, 'Sugarland', 5),
    ('ProductZ', 3, 'Houston', 5),
    ('Computerization', 10, 'Stafford', 4),
    ('Reorganization', 20, 'Houston', 1),
    ('Newbenefits', 30, 'Stafford', 4);

INSERT INTO WORKS_ON VALUES
    ('123456789', 1, 32.5),
    ('123456789', 2, 7.5),
    ('666884444', 3, 40.0),
    ('453453453', 1, 20.0),
    ('453453453', 2, 20.0),
    ('333445555', 2, 10.0),
    ('333445555', 3, 10.0),
    ('333445555', 10, 10.0),
    ('333445555', 20, 10.0),
    ('999887777', 30, 30.0),
    ('999887777', 10, 10.0),
    ('987987987', 10, 35.0),
    ('987987987', 30, 5.0),
    ('987654321', 30, 20.0),
    ('987654321', 20, 15.0),
    ('888665555', 20, NULL);

INSERT INTO DEPENDENT VALUES
    ('333445555', 'Alice', 'F', '1986-04-05', 'DAUGHTER'),
    ('333445555', 'Theodore', 'M', '1983-10-25', 'SON'),
    ('333445555', 'Joy', 'F', '1958-05-03', 'SPOUSE'),
    ('987654321', 'Abner', 'M', '1942-02-28', 'SPOUSE'),
    ('123456789', 'Michael', 'M', '1988-01-04', 'SON'),
    ('123456789', 'Alice', 'F', '1988-12-30', 'DAUGHTER'),
    ('123456789', 'Elizabeth', 'F', '1967-05-05', 'SPOUSE');
\end{lstlisting}

\subsection{VIEW}

\begin{lstlisting}[language=sql]
USE COMPANY;
GO

-- a. A view that has the department name, manager name, and manager salary for every department.
CREATE VIEW VIEW_A
AS
    SELECT DNAME, FNAME, MINIT, LNAME, SALARY
    FROM DEPARTMENT JOIN EMPLOYEE ON MGR_SSN = SSN;
GO

-- b. A view that has the employee name, supervisor name, and employee salary for each employee who works in the 'Research' department. 
CREATE VIEW VIEW_B
AS
    SELECT
        E1.FNAME AS EMPLOYEE_FNAME,
        E1.MINIT AS EMPLOYEE_MINIT,
        E1.LNAME AS EMPLOYEE_LNAME,
        E2.FNAME AS SUPER_FNAME,
        E2.MINIT AS SUPER_MINIT,
        E2.LNAME AS SUPER_LNAME,
        E1.SALARY
    FROM (EMPLOYEE E1 JOIN EMPLOYEE E2 ON E1.SUPER_SSN = E2.SSN) JOIN DEPARTMENT D1 ON E1.DNO = D1.DNUMBER
    WHERE D1.DNAME = 'Research';
GO

-- c. A view that has the project name, controlling department name, number of employees, and total hours worked per week on the project for each project.
CREATE VIEW VIEW_C
AS
    SELECT
        PNAME,
        DNAME, COUNT(ESSN) AS NUM_EMPLOYEE,
        SUM([HOURS]) AS TOTAL_HOUR
    FROM
        (PROJECT JOIN DEPARTMENT ON DNUM = DNUMBER) JOIN WORKS_ON ON PNUMBER = PNO
    GROUP BY PNAME, DNAME;
GO

-- d. A view that has the project name, controlling department name, number of employees, and total hours worked per week on the project for each project with more than two employees working on it. 
CREATE VIEW VIEW_D
AS
    SELECT
        PNAME,
        DNAME, COUNT(ESSN) AS NUM_EMPLOYEE,
        SUM([HOURS]) AS TOTAL_HOUR
    FROM
        (PROJECT JOIN DEPARTMENT ON DNUM = DNUMBER) JOIN WORKS_ON ON PNUMBER = PNO
    GROUP BY PNAME, DNAME
    HAVING COUNT(ESSN) > 2;
GO

-- e. A view (SSN, Full Name of employee, Number of dependents) that includes information about employees who have the number of dependents greater than 2. 
CREATE VIEW VIEW_E
AS
    SELECT
        SSN,
        CONCAT(FNAME, ' ', MINIT, ' ', LNAME) AS FULL_NAME,
        COUNT(DEPENDENT_NAME) AS NUM_DEPENDENT
    FROM EMPLOYEE JOIN DEPENDENT ON SSN = ESSN
    GROUP BY SSN, FNAME, MINIT, LNAME
    HAVING COUNT(DEPENDENT_NAME) > 2; 
GO

-- f. A view (Full Name of employee, date of birth, gender) for those employees who have their birthdate in July. 
CREATE VIEW VIEW_F
AS
    SELECT
        CONCAT(FNAME, ' ', MINIT, ' ', LNAME) AS FULL_NAME,
        BDATE,
        SEX
    FROM EMPLOYEE
    WHERE MONTH(BDATE) = 7;
GO

-- g. A view (Name of dependent, SSN of employee, date of birth of dependent) that includes information on all dependents who are less than 18 years old. 
CREATE VIEW VIEW_G
AS
    SELECT DEPENDENT_NAME, ESSN, BDATE
    FROM DEPENDENT
    WHERE BDATE > DATEADD(YEAR, -18, GETDATE());
GO
\end{lstlisting}

\textbf{Output results:}
See figure \ref{fig:view_a}, \ref{fig:view_b}, \ref{fig:view_c}, \ref{fig:view_d}, \ref{fig:view_e}, \ref{fig:view_f}, \ref{fig:view_g}.

\begin{figure}[!h]
    \centering
    \includegraphics[width=.75\linewidth]{image/1.1.a.png}
    \caption{Result of VIEW\_A}
    \label{fig:view_a}
\end{figure}

\begin{figure}[!h]
    \centering
    \includegraphics[width=.75\linewidth]{image/1.1.b.png}
    \caption{Result of VIEW\_B}
    \label{fig:view_b}
\end{figure}

\begin{figure}[!h]
    \centering
    \includegraphics[width=.75\linewidth]{image/1.1.c.png}
    \caption{Result of VIEW\_C}
    \label{fig:view_c}
\end{figure}

\begin{figure}[!h]
    \centering
    \includegraphics[width=.75\linewidth]{image/1.1.d.png}
    \caption{Result of VIEW\_D}
    \label{fig:view_d}
\end{figure}

\begin{figure}[!h]
    \centering
    \includegraphics[width=.75\linewidth]{image/1.1.e.png}
    \caption{Result of VIEW\_E}
    \label{fig:view_e}
\end{figure}

\begin{figure}[!h]
    \centering
    \includegraphics[width=.75\linewidth]{image/1.1.f.png}
    \caption{Result of VIEW\_F}
    \label{fig:view_f}
\end{figure}

\begin{figure}[!h]
    \centering
    \includegraphics[width=.75\linewidth]{image/1.1.g.png}
    \caption{Result of VIEW\_G}
    \label{fig:view_g}
\end{figure}